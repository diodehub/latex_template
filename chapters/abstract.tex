\begin{abstract}
\textbf{ABSTRACT} \\
The purpose of this study was to investigate the effects of exercise on cognitive function in older adults. A randomized controlled trial was conducted with 100 participants aged 65 and over, who were assigned to either a control group or an exercise group. The exercise group participated in a supervised aerobic and resistance training program three times per week for 12 weeks, while the control group did not participate in any structured exercise. Cognitive function was measured using a battery of standard neuropsychological tests at baseline and after the intervention. Results showed that the exercise group had significantly improved scores on measures of executive function, processing speed, and working memory compared to the control group. These findings suggest that regular exercise may have a positive impact on cognitive function in older adults.
\end{abstract}
